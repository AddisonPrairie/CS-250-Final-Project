\documentclass{article}


\usepackage{preamble}

\title{Error Correcting Codes, Hardness Amplification and Boosting.}
\date{}

\begin{document}
    
\noindent Error Correcting Codes, Hardness Amplification and Boosting \hfill  CS 250, Winter 2025\\
\hrule

\section{Introduction}

\section{Preliminaries}

\section{Hardness Amplification with ECCs}

\begin{definition}{3.x}
    We say that a function $G : \pmo^s \rightarrow \pmo^{n}$ is an $(s, n)$-PRG if, for any size $n$ circuit $C$ over $n$ inputs, 
    \begin{equation*}
        \left|\Pr_{x \backsim U_n}[C(x)] - \Pr_{x \backsim U_s}[C(G(s))]\right| < \frac{1}{10}.
    \end{equation*}
\end{definition}

The parameter $s$ is the seed length.

\subsection{Why Hardness Amplification?}

\begin{theorem}{3.x} Suppose that $G$ is a $(O(\log n), n)$-PRG computable in $\poly(n)$ time. Then $\P = \BPP$.
\end{theorem}

\subsubsection{A Simple PRG}

\begin{theorem}{3.x}
    Given the average case hardness assumption, there exists a $(n - 1, n)$-PRG.
\end{theorem}

\subsubsection{The Nisan-Wigderson Generator and BPP}

\subsection{Worst to Average Case Hardness}

\section{Boosting and Hardness Amplification}

\subsection{Smooth Boosters and The Hardcore Theorem}

\subsection{Boosting via the XOR Lemma}

\section{One Shot Boosting with ECCs}

\begin{theorem}{1.2.3}\label{t-1.2.3}
    Theorem 1.2
\end{theorem}
    
\begin{claim}{1.2}

\end{claim}

\begin{observation}{1.5}
\end{observation}

\begin{lemma}{1.5}
\end{lemma}

\begin{definition}{1.5}
\end{definition}

Note that \hyperref[t-1.2.3]{Theorem 1.2.3}

\end{document}